\documentclass[14pt,xcolor=dvipsnames]{beamer}
\usepackage[utf8x]{inputenc}
\usepackage[T1]{fontenc}
\usepackage[ngerman]{babel}
\usepackage{hyperref}
%\usepackage[svgnames]{xcolor}
%\usetheme[secheader]{Boadilla}
%\usefonttheme{sans}
\setbeamersize{text margin left=1cm,text margin right=1cm}

\title{Die Burrows-Wheeler-Transformation}
\subtitle{Proseminar \glqq Algorithmen der Bioinformatik\grqq}
\author{Uli Köhler}
\date{12.~November 2012}

%Colors
\definecolor{darkgreen}{rgb}{136,100,62}

\AtBeginSection[]{} % for optional outline or other recurrent slide

\begin{document}
\frame{\titlepage}
\begin{frame}
\frametitle{Aufbau dieser Präsentation}
\tableofcontents
\end{frame}

\section{Kernkonzepte}
\subsection{Burrows-Wheeler-Transformation}
%%%%%% 	
\begin{frame}[allowframebreaks]
 \frametitle{BWT - Kompression}
    \begin{columns}[c,onlytextwidth]
    \column{0.55\textwidth}
    \begin{itemize}
	\item Eingabestring: \textit{aabrac}
	\item Bildung aller zyklischen Rotationen des Eingabestrings
    \end{itemize}
    \column{0.45\textwidth}
    \begin{tabular}{c|c}
    \textbf{Row \#} & \textbf{Rotation} \\
    0 & aabrac \\
    1 & abraca \\
    2 & bracaa \\
    3 & racaab \\
    4 & acaabr \\
    5 & caabra \\
    \end{tabular}
    \end{columns}
\end{frame}
%%%%%% Sorting
\begin{frame}
\begin{columns}[c,onlytextwidth]
 \column{.55\textwidth}
 \begin{itemize}
  \item Sortieren der Rotationen des Eingabestrings
 \end{itemize}
 \column{.45\textwidth}
    \begin{tabular}{c|c}
    \textbf{Row \#} & \textbf{Rotation} \\
    0 & aabrac \\
    1 & abraca \\
    2 & acaabr \\
    3 & bracaa \\
    4 & caabra \\
    5 & racaab \\
    \end{tabular}
\end{columns}
\end{frame}
%%% Output of compression algorithm
\begin{frame}
\begin{columns}[c,onlytextwidth]
 \column{.55\textwidth}
 \begin{itemize}
  \item Resultat:\\
      Das Tupel \textit{$(L,I)$}, wobei L die {\color{red}letzte Spalte} der Matrix ist
      und I der Index des {\color{darkgreen}Eingabestrings in der Matrix} ist
  \item $(L,I) = (caraab, 1)$
 \end{itemize}
 \column{.45\textwidth}
    \begin{tabular}{c|cc}
    \textbf{Row \#} & \textbf{Rotation} \\
    0 & aabra{\color{red}c} \\
    {\color{darkgreen}1} & {\color{darkgreen}abrac}{\color{red}a} \\
    2 & acaab{\color{red}r} \\
    3 & braca{\color{red}a} \\
    4 & caabr{\color{red}a} \\
    5 & racaa{\color{red}b} \\
    \end{tabular}
\end{columns}
\end{frame}
%%%%%% Decompression
\begin{frame}

\begin{columns}[c,onlytextwidth]
 \column{.55\textwidth}
 \begin{itemize}
  \item Resultat:\\
      Das Tupel \textit{$(L,I)$}, wobei L die {\color{red}letzte Spalte} der Matrix ist
      und I der Index des {\color{darkgreen}Eingabestrings in der Matrix} ist
  \item $(L,I) = (caraab, 1)$
 \end{itemize}
 \column{.45\textwidth}
    \begin{tabular}{c|cc}
    \textbf{Row \#} & \textbf{Rotation} \\
    0 & aabra{\color{red}c} \\
    {\color{darkgreen}1} & {\color{darkgreen}abrac}{\color{red}a} \\
    2 & acaab{\color{red}r} \\
    3 & braca{\color{red}a} \\
    4 & caabr{\color{red}a} \\
    5 & racaa{\color{red}b} \\
    \end{tabular}
\end{columns}
\end{frame}
%%%%%% Complexity
\begin{frame}
 \frametitle{Komplexität der BWT}
 \begin{itemize}
  \item Die Komplexität der BWT ist mindestens $\mathcal{O}(n^2)$
  %TODO 
 \end{itemize}
\end{frame}
\end{document}
